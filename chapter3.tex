\chapter{Protocols}

\clearpage
\section{Password Eavesdropping Risks}
A good case study comes from simple embedded systems, such as the remote control used to open
your garage or to unlock the doors of cars manufactured up to the mid-1990’s.
These primitive remote controls just broadcast their serial number, which also
acts as the password. An attack that became common was to use a ‘grabber’, a device 
that would record a code broadcast locally and replay it later. 

sixteen-bit passwords are too short. It occasionally happened that people found they could unlock 
the wrong car by mistake. By the mid-1990’s, devices appeared which could try all
possible codes one after the other. A code will be found on average after about
215 tries, which at ten per second takes under an hour. A thief operating in a
parking lot with a hundred vehicles within range would be rewarded in less
than a minute with a car helpfully flashing its lights.



\section{Simple Authentication}

	{\bf Nonce:} The term nonce can mean anything that guarantees the freshness of a
	message. A nonce can, according to the context, be a random number, a serial
	number, a random challenge received from a third party, or even a timestamp.

	{\bf Security and business: }
	Security mechanisms are used more and more to support business 
	models, by accessory control, rights management, product tying and bundling. It is
	wrong to assume blindly that security protocols exist to keep ‘bad’ guys ‘out’.
	They are increasingly used to constrain the lawful owner of the equipment in
	which they are built; their purpose may be of questionable legality or contrary
	to public policy. For example: Many printer companies embed authentication mechanisms 
	in printers to ensure that genuine toner cartridges are used. If a competitor’s 
	product is loaded instead, the printer may quietly downgrade from 1200 dpi to 300 dpi, 
	or simply refuse to work at all. Mobile phone vendors make a lot of money from 
	replacement batteries, and now use authentication protocols to spot competitors’ 
	products so they can be blocked or even drained more quickly. 

	\clearpage
	\subsection{Challenge and response}

		Many cars use a more sophisticated two-pass protocol, called challenge-response, 
		to actually authorise engine start. As the car key is inserted into the steering lock, 
		the engine controller sends a challenge consisting of a random n-bit number to the key 
		using short-range radio. The car key computes a response by encrypting the challenge. 
		This is still not bulletproof.


	\subsection{Reflection Attacks}
		

\section{Manipulating the Message}

\section{Changing the Environment}

\section{Chosen Protocol Attacks}

\section{Managing Encryption Keys}

	\subsection{Basic Key Management}

	\subsection{The Needham-Schroeder Protocol}

	\subsection{Kerberos}

	\subsection{Practical Key Management}

\section{Getting Formal}

	\subsection{A Typical Smartcard Banking Protocol}

	\subsection{The BAN Logic}



