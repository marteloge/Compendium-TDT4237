\chapter{Web Application Penetration Testing}

	\section{Introduction to Penetration Testing}

	\subsection*{What is Web Application Penetration Testing?}

		A penetration test is a method of evaluating the security of a computer system or network 
		by simulating an attack. A Web Application Penetration Test focuses only on evaluating the 
		security of a web application. The process involves an active analysis of the application 
		for any weaknesses, technical flaws, or vulnerabilities. 

	\subsection*{What is a vulnerability?}
		{\bf A vulnerability} is a flaw or weakness in a system's design, implementation, or operation 
		and management that could be exploited to violate the system's security policy. 
		
		{\bf A threat} is a potential attack that, by exploiting a vulnerability, may harm the assets 
		owned by an application (resources of value, such as the data in a database or in the file system). 

		{\bf A test} is an action that tends to show a vulnerability in the application.

	\subsection*{What is the OWASP testing methodology?}
		{\bf Penetration testing} will never be an exact science where a complete list of all possible 
		issues that should be tested can be defined. Indeed, penetration testing is only an appropriate
		technique for testing the security of web applications under certain circumstances. 

		{\bf The goal} is to collect all the possible testing techniques, explain them and keep the 
		guide updated. The OWASP Web Application Penetration Testing method is based on the {\bf black box}
		approach. The tester knows nothing or very little information about the application to be tested. 

		{\bf The testing model consists of:}
			\begin{itemize}
				\item {\bf Tester:} Who performs the testing activities
				\item {\bf Tools and methodology:} The core of this Testing Guide project
				\item {\bf Application:} The black box to test
			\end{itemize}

	\subsection*{Active and Passive Mode}

		{\bf Passive Mode:} in the passive mode, the tester tries to understand the application's logic, 
		and plays with the application. Tools can be used for information gathering, for example, an 
		HTTP proxy to observe all the HTTP requests and responses. At the end of this phase, the tester 
		should understand all the access points (gates) of the application (e.g., HTTP headers, parameters,
		and cookies). The Information Gathering section explains how to perform a passive mode test.

		{\bf Ative Mode:}  in this phase, the tester begins to test (the penetration testing) using 
		the methodology described in the specific penetration category.
