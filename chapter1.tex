\chapter{What is Security Engeneering?}
\clearpage

{\bf A Framework} Good security engeneering requires four things to come together:
	\begin{enumerate} 
		\item {\bf Policy:} what you're supposed to achieve.
		\item {\bf Mechanism:} the chipers, access controls, hardware tamper-resistance
		and other mechinery that you assemblein order to implement the policy.
		\item {\bf Assurance:} the amount of reliance you can place on each particular mechanism.
		\item {\bf Incentive:} the motive that people guarding and maintaining the system
		have to do their job properly, and also the motive that the attackers have to try to defeat
		your policy. 
	\end{enumerate}


{\bf Why ae poor policy choices made? } Quite simply, the incentives on the decision makers
favor visible controls over effective ones. The result is what Bruce Schneier calls 
"security theatre" - measures designed to produce a feeling of security rather than the reality.


{\bf The role as a security engineer:} 
	\begin{itemize}
		\item We need to be able to put risks and threats in content.
		\item We need to be able to make realistic assessments of what might go wrong.
		\item We need to give our clients good advice.
	\end{itemize}


{\bf Case studies:} I will not describe these in detail. It is described in the book (pages 6-11).
	\begin{itemize}
		\item A bank
		\item A military base
		\item A hospital
		\item The Home
	\end{itemize}

\clearpage
\section{Definitions}
\subsection{System}
	\begin{enumerate}
		\item a product or component, such as a cryptographic protocol, a smartcard
		or the hardware of a PC.
		\item a collection of the above plus an operating system, communications and
		other things that go to make up an organization’s infrastructure.
		\item the above plus one or more applications (media player, browser, word
		processor, accounts / payroll package, and so on).
		\item any or all of the above plus IT staff.
		\item any or all of the above plus internal users and management;
		\item any or all of the above plus customers and other external users
	\end{enumerate}

\subsection{Subject, Person and Principal}
	\begin{itemize}
		\item{\bf Subject:} By a subject I will mean a physical person (human, ET, ...), 
		in any role including that of an operator, principal or victim. 
		\item{\bf Person:} By a person, I will mean either a physical person or a 
		legal person such as a company or government.
		\item{\bf Principal:} A principal is an entity that participates in a security system. 
		This entity can be a subject, a person, a role, or a piece of equipment such as a PC, smartcard, or
		card reader terminal. A principal can also be a communications channel (which
		might be a port number, or a crypto key, depending on the circumstance). A
		principal can also be a compound of other principals
	\end{itemize}

\subsection{Identity} 

\clearpage
\subsection{Trust and Trustworthy} 
	The definitions of trust and trustworthy are often confused. 
	The following example illustrates the difference: if an NSA employee is observed in a toilet
	stall at Baltimore Washington International airport selling key material to a
	Chinese diplomat, then (assuming his operation was not authorized) we can
	describe him as ‘trusted but not trustworthy’. Hereafter, we’ll use the NSA
	definition that a trusted system or component is one whose failure can break the
	security policy, while a trustworthy system or component is one that won’t fail.

\subsection{Confidenciality, Privacy and Secrecy}
	\begin{itemize}
		\item {\bf Secrecy} is a technical term which refers to the effect of the mechanisms
		used to limit the number of principals who can access information, such
		as cryptography or computer access controls.
		\item {\bf Confidentiality} involves an obligation to protect some other person’s 
		or organization’s secrets if you know them.
		\item {\bf Privacy} is the ability and/or right to protect your personal information
		and extends to the ability and/or right to prevent invasions of personal space.
	\end{itemize}

\subsection{Authenticity and Integrity}


\clearpage
\section*{Summary}
\begin{itemize}
	\item Framework
	\item Security theatre
	\item The role as a security engineer
	\item Case studies (bank, military base, hospital, the Home)
	\item Definition: System
	\item Definitions: subject, person and principal
	\item Identity
	\item Trust and Trustworthy
	\item Confidenciality, Privacy and Secrecy
	\item Authenticity and Integrity
\end{itemize}

 